\documentclass[11pt]{article}
\usepackage{ifthen}
\usepackage{graphicx}
\usepackage{cite}
\usepackage{makeidx}
\usepackage{hevea}
\input{../../synrc.hva}
%HEVEA \loadcssfile{../../synrc.css}

\begin{document}

\title{N2O: API}
\author{Maxim Sokhatsky}

%HEVEA \begin{divstyle}{nonselectedwrapper}
%HEVEA \begin{divstyle}{article}
%HEVEA \begin{divstyle}{smallcol}
\paragraph{}
\includeimage{../../images/n2o.png}
%HEVEA \end{divstyle}
%HEVEA \begin{divstyle}{articlecol}
\section*{N2O: Core API}

\subsection*{Update DOM \bf{wf:update}}
Issue #update{} action for an element.
It generates jQuery DOM update script and evaluates it.

\subsection*{Wire JavaScript \bf{wf:wire}}
Issue #wire{} action that sends JavaScript for evaluation on the client.

\subsection*{Spawn Async Processes \bf{wf:comet}}
Creates Erlang processes that talks to primary page process by sending messages.
For redirect all updates and wire actions to page process {\bf wf:flush} should be called.

\subsection*{Parse URL and Context parameters \bf{wf:q}}
For extraction url parameters or reads from process context.

\subsection*{Redirect to or send a new page \bf{wf:redirect}}
For login redirecting use {\bf{wf:redirect_to_login}} and {\bf{wf:redirect_from_login}}.
It saves login context information and sends it in first packet after establishing WebSocket connection.

\subsection*{GProc process registration {\bf wf:reg} and {\bf wf:send}}
For managing pools of async processes N2O uses GProc process registry.
Now you can control allocation of processes yourself,
and {\bf wf:comet\_global} and {\bf wf:send_global} are obsolete.
You can assign a process to the pool with {\bf wf:reg}.

%HEVEA \rawhtmlinput{templates/disqus.htx}

%HEVEA \end{divstyle}

%HEVEA \begin{divstyle}{toc last}

\section*{TOC}
\paragraph{}
\footahref{http://synrc.com/framework/web/}{Overview} \@br
\footahref{http://synrc.com/framework/web/setup.htm}{1. Setup} \@br
\footahref{http://synrc.com/framework/web/elements.htm}{2. Elements} \@br
\footahref{http://synrc.com/framework/web/actions.htm}{3. Actions} \@br
\footahref{http://synrc.com/framework/web/api.htm}{4. Core API} \@br
\footahref{http://synrc.com/framework/web/extending.htm}{5. Extending} \@br
% \footahref{http://synrc.com/framework/web/routing.htm}{6. Routing} \@br
% \footahref{http://synrc.com/framework/web/samples.htm}{5. Samples} \@br


%HEVEA \end{divstyle}

%HEVEA \end{divstyle}
%HEVEA \end{divstyle}
%HEVEA \begin{divstyle}{clear}{~}\end{divstyle}

\end{document}

