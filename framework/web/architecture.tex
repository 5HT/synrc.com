\documentclass[11pt]{article}
\usepackage{ifthen}
\usepackage{graphicx}
\usepackage{cite}
\usepackage{hevea}
\input{../../synrc.hva}
%HEVEA \loadcssfile{../../synrc.css}

\begin{document}

\title{N2O: Architecture}
\author{Maxim Sokhatsky}

%HEVEA \begin{divstyle}{nonselectedwrapper}
%HEVEA \begin{divstyle}{article}
%HEVEA \begin{divstyle}{smallcol}
\paragraph{}
\includeimage{../../images/n2o.png}
%HEVEA \end{divstyle}
%HEVEA \begin{divstyle}{articlecol}
\section*{N2O: Architecture}

\subsection*{Reduced Latency}
The secret of reduced latency is simple. We try to deliver rendered HTML
as soon as possible and render JavaScript only after WebSocket initialization.
We use thre steps and three erlang processes for achieve that.

%HEVEA \rawhtmlinput{templates/page-lifetime.htx}

\subsection*{HTTP process}
In first HTTP handler we render only HTML and all created by
the way action is stored in created transition process.

%HEVEA \rawhtmlinput{templates/transition.htx}

HTTP handler dies immediately after terurning HTML. Transition
process waits for retrival request from future WebSocket handler.

\subsection*{Transition process}
Just after receiving HTML browser initiates WebSocket connection
and WebSocket handler arise. After returning actions transition
process dies and from now on WebSocket handler stay alone.
So initial phase done.

\subsection*{WebSocket process}
After that through WebSocket channel all event comes from
browser to server and handler by N2O, who renders elements
to HTML and actions to JavaScript.

%HEVEA \rawhtmlinput{templates/disqus.htx}

%HEVEA \end{divstyle}

%HEVEA \begin{divstyle}{toc last}

\section*{TOC}
\paragraph{}
\footahref{http://synrc.com/framework/web/}{Overview} \@br
\footahref{http://synrc.com/framework/web/setup.htm}{1. Setup} \@br
\footahref{http://synrc.com/framework/web/elements.htm}{2. Elements} \@br
\footahref{http://synrc.com/framework/web/actions.htm}{3. Actions} \@br
\footahref{http://synrc.com/framework/web/api.htm}{4. Core API} \@br
\footahref{http://synrc.com/framework/web/extending.htm}{5. Extending} \@br
% \footahref{http://synrc.com/framework/web/routing.htm}{6. Routing} \@br
% \footahref{http://synrc.com/framework/web/samples.htm}{5. Samples} \@br


%HEVEA \end{divstyle}

%HEVEA \end{divstyle}
%HEVEA \end{divstyle}
%HEVEA \begin{divstyle}{clear}{~}\end{divstyle}

\end{document}

