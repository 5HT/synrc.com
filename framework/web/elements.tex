\documentclass[11pt]{article}
\usepackage{ifthen}
\usepackage{graphicx}
\usepackage{cite}
\usepackage{hevea}
\input{../../synrc.hva}
%HEVEA \loadcssfile{../../synrc.css}

\begin{document}

\title{N2O: Elements}
\author{Maxim Sokhatsky}

%HEVEA \begin{divstyle}{nonselectedwrapper}
%HEVEA \begin{divstyle}{article}
%HEVEA \begin{divstyle}{smallcol}
\paragraph{}
\includeimage{../../images/n2o.png}
%HEVEA \end{divstyle}
%HEVEA \begin{divstyle}{articlecol}
\section*{N2O: Erlang DSL for HTML Elements}

\subsection*{Page construction}
In N2O you usually don't use HTML at all. Instead you provide page definition
in form of Erlang records and thus page is typed and checked on compilation time.
This is a classical CGI approach for compiled pages which give us benefits like
compilation-time error-checking and great performance.

\subsection*{Core Elements}
Core set of HTML elements includes br, headings, links, tables, lists and image.

\subsection*{Form Elements}
Standard active elements which provide some information to server
and gather user input are button, radio and check buttons, text box and area and password box.

\subsection*{Advanced Elements}
Please refer to \footahref{http://synrc.com/framework/web/extending.htm}{Chapter 5. Extending N2O Elements}.

%HEVEA \rawhtmlinput{templates/disqus.htx}

%HEVEA \end{divstyle}

%HEVEA \begin{divstyle}{toc last}

\section*{TOC}
\paragraph{}
\footahref{http://synrc.com/framework/web/}{Overview} \@br
\footahref{http://synrc.com/framework/web/setup.htm}{1. Setup} \@br
\footahref{http://synrc.com/framework/web/elements.htm}{2. Elements} \@br
\footahref{http://synrc.com/framework/web/actions.htm}{3. Actions} \@br
\footahref{http://synrc.com/framework/web/api.htm}{4. Core API} \@br
\footahref{http://synrc.com/framework/web/extending.htm}{5. Extending} \@br
% \footahref{http://synrc.com/framework/web/routing.htm}{6. Routing} \@br
% \footahref{http://synrc.com/framework/web/samples.htm}{5. Samples} \@br


%HEVEA \end{divstyle}

%HEVEA \end{divstyle}
%HEVEA \end{divstyle}
%HEVEA \begin{divstyle}{clear}{~}\end{divstyle}

\end{document}

