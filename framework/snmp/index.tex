\documentclass[11pt]{article}
\usepackage{ifthen}
\usepackage{graphicx}
\usepackage{cite}
\usepackage{hevea}
\input{synrc.hva}
%HEVEA \loadcssfile{../../synrc.css}

\begin{document}

\title{SNMP}
\author{Maxim Sokhatsky}

%HEVEA \begin{divstyle}{nonselectedwrapper}
%HEVEA \begin{divstyle}{article}
%HEVEA \begin{divstyle}{smallcol}
\includeimage{../../images/mandalalogo.png}
%HEVEA \end{divstyle}
%HEVEA \begin{divstyle}{articlecol}
\section*{Mandala: SNMP Monitoring for Erlang cloud}
\subsection*{Simple way to understand your system}
We rethink the way system should be managed. We strongly believe that
SNMP monitoring system should be clear and understandable to customer.
We try to avoid complex multi-window solutions with bunch of optional information.

\subsection*{Mandala: Load Tracking}
%HEVEA \rawhtmlinput{templates/mandala.htx}

Each sector shows the state of Erlang VM. Colors are Erlang release types.
Arcs are the release lines. You can see the performance and memory
consumption of each line and each server.

\subsection*{HAL 9000: Tracking Events}
%HEVEA \rawhtmlinput{templates/hal.htx}

HAL 9000 inspired tiles show the detailed state of server in changable
order according to scheduling priorities where alert notification are more
important.

\subsection*{Injectable SNMP Agents in Erlang}
We provide clean toolset and SNMP agent skeletons for injection into existing Erlang
solutions such as RabbitMQ, Riak, etc.
%HEVEA \end{divstyle}

%HEVEA \begin{divstyle}{toc last}
\section*{TOC}
\paragraph{}
\footahref{}{Overview} \@br
\footahref{}{1. Mandala} \@br
\footahref{}{2. HAL 9000} \@br
\footahref{}{3. Modules} \@br
\footahref{}{4. Charts} \@br
%HEVEA \end{divstyle}

%HEVEA \end{divstyle}
%HEVEA \end{divstyle}
%HEVEA \begin{divstyle}{clear}{~}\end{divstyle}

\end{document}

