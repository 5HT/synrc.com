\documentclass[11pt]{article}
\usepackage{ifthen}
\usepackage{graphicx}
\usepackage{cite}
\usepackage{hevea}
\input{synrc.hva}
%HEVEA \loadcssfile{synrc.css}
\begin{document}

\title{Application Stack}
\author{Maxim Sokhatsky}

%HEVEA \begin{divstyle}{selectedwrapper}
%HEVEA \begin{divstyle}{wrapper}

%HEVEA \begin{divstyle}{block}
\section*{KVS}
\paragraph{}
KVS is Key-Value Storage Data Framework that provide high-level API for handling
social data like Users, Feeds, Comments, etc. It is designed with sequential consistency in mind.

\footahref{https://synrc.github.io/kvs}{synrc.github.io/kvs} ver 2.8
%HEVEA \end{divstyle}


%HEVEA \begin{divstyle}{block}
\section*{\footahref{http://synrc.com/apps/n2o}{N2O}}

\paragraph{}
Most Powerful Erlang WebSocket Application Server. Server-render and SPA modes.
Extensible WebSocket protocol for JSON, BERT, RAW binary data transfer.
Includes: REST, JavaScript Compiler, Pub/Sub, Sessions, Controls, Templates, etc.

\footahref{https://synrc.github.io/n2o}{synrc.github.io/n2o} ver 2.8
%HEVEA \end{divstyle}

%HEVEA \begin{divstyle}{block}
\section*{\footahref{http://synrc.com/apps/mad}{MAD}}

\paragraph{}
Mad is Erlang Dependency Manager that aims to be small and fast rebar
replacement which supports rebar.config files, YRL and DTL compilation.

\footahref{https://synrc.github.io/mad}{synrc.github.io/mad} ver 1.4
%HEVEA \end{divstyle}


%HEVEA \begin{divstyle}{block}
\section*{\footahref{http://synrc.com/apps/upl}{UPL}}
\paragraph{}
The foundation of Spawnproc is strictly typed banking specific
programming language, which has loans/deposits, cashbacks,
grace periods as its primitives, all defined in clean manner.

\footahref{https://spawnproc.github.io/upl/}{spawnproc.github.io/upl}
%HEVEA \end{divstyle}

%HEVEA \begin{divstyle}{block}
\section*{\footahref{http://synrc.com/apps/bpe}{BPE}}
\paragraph{}
The engine defines business flows according to industry standards.
Each flow is accompanied by all relevant documents, such as
approval forms, sing-off sheets, legal agreements.

\footahref{https://spawnproc.github.io/forms/}{spawnproc.github.io/bpe} ver 0.7
%HEVEA \end{divstyle}

%HEVEA \begin{divstyle}{block}
\section*{\footahref{http://synrc.com/apps/forms}{FORMS}}
\paragraph{}
We automatically generates the visual forms for data assessment
based on ontological model. These forms are fulfilled with native
data for KVS storage.

\footahref{https://spawnproc.github.io/forms/}{spawnproc.github.io/forms} ver 0.7
%HEVEA \end{divstyle}

%HEVEA \begin{divstyle}{block}
\section*{SHEN}
\paragraph{}    
JavaScript parse transform allows you to write JavaScript in Erlang and
compile it with erlc. This compiler preserves program semantics closure to closure.

\footahref{https://synrc.github.io/shen}{synrc.github.io/shen} ver 1.5
%HEVEA \end{divstyle}

%HEVEA \begin{divstyle}{block}
\section*{REST}
\paragraph{}
REST toolkit that allows you to deal with typed JSON as Erlang records.
It will automatically generates JSON/Records converters for well knwon records.
It is released as standalone micro-REST app that could be used with Cowboy.

\footahref{https://synrc.github.io/rest}{synrc.github.io/rest} ver 2.6
%HEVEA \end{divstyle}


%HEVEA \begin{divstyle}{block}
\section*{ACTIVE}
\paragraph{}
Active is sync replacement that uses native FileSystem OS async
listeners to compile and reload Erlang files, DTL templates and other files.
It acts as FS subscriber under supervision and uses MAD under the hood.

\footahref{https://synrc.github.io/active}{synrc.github.io/active} ver 1.4
%HEVEA \end{divstyle}

%HEVEA \begin{divstyle}{block}
\section*{FS}
\paragraph{}
File System listener provides native async way of watching on
file system changes unlike polling in Nitrogen's sync. It will save
your CPU in production.

\footahref{https://synrc.github.io/fs}{synrc.github.io/fs} ver 1.4
%HEVEA \end{divstyle}

%HEVEA \begin{divstyle}{block}
\section*{SH}
\paragraph{}
Erlang Shell Executor will give you safe access to calling system shell
from erlang and also prevent your erlang processes from leaking on port closing.

\footahref{https://synrc.github.io/sh}{synrc.github.io/sh} ver 1.4
%HEVEA \end{divstyle}

%HEVEA \begin{divstyle}{block}
\section*{AVZ}
\paragraph{}
AVZ provides simple and sane API for JavaScript based and HTTP-redirect
based auth methods like Google, Facebook, Microsoft, Twitter and Github.
It is very tiny, clean and useful for your land pages.

\footahref{https://synrc.github.io/avz}{synrc.github.io/avz} ver 2.1
%HEVEA \end{divstyle}


%HEVEA \begin{divstyle}{block}
\section*{OTP.MK}
\paragraph{}

Tiny Makefile-based Erlang/OTP and reltool/relx/rebar/mix/mad compatible
build solutions. Today otp.mk costs us 52 LOC and orderapps.erl 15 LOC
and we want to keep that size. Consider this as top level Makefile-base
API for Erlang tools.

\footahref{https://synrc.github.io/otp.mk}{synrc.github.io/otp.mk} ver 1.5
%HEVEA \end{divstyle}

%HEVEA \begin{divstyle}{block}
\section*{FEEDS}
\paragraph{}
Feed Server is node of user workers region. It handles all MQ
requests for write operations for user's data and other APIs.
Also it acts as distributed cache for user feeds and other list chains.

\footahref{https://synrc.github.io/feeds/}{synrc.github.io/feeds}
%HEVEA \end{divstyle}


%HEVEA \begin{divstyle}{block}
\section*{MQS}
\paragraph{}
MQS is RabbitMQ client library that handles conections, channels,
subscriptions has its own routing interface that you can use for building
sophisticated subscriptions topology. It also supports RPC over MQ.

\footahref{https://synrc.github.io/mqs/}{synrc.github.io/mqs} ver 0.4
\vspace{2\baselineskip}

%HEVEA \end{divstyle}

%HEVEA \end{divstyle}
%HEVEA \end{divstyle}

%HEVEA \begin{divstyle}{nonselectedwrapper}
%HEVEA \begin{divstyle}{verywidecol}

%HEVEA \rawhtmlinput{templates/depot.htx}

%HEVEA \end{divstyle}
%HEVEA \end{divstyle}


%HEVEA \begin{divstyle}{clear}{~}\end{divstyle}

%HEVEA \begin{divstyle}{nonselectedwrapper}
%HEVEA \begin{divstyle}{verywidecol}

\footahref{http://synrc.com/feedback.htm}{Contact Us}\qquad

%HEVEA \end{divstyle}
%HEVEA \end{divstyle}

\end{document}
