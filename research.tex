\documentclass[11pt]{article}
\usepackage{ifthen}
\usepackage{graphicx}
\usepackage{cite}
\usepackage{hevea}
\input{synrc.hva}
%HEVEA \loadcssfile{synrc.css}
\begin{document}

\title{Showcase}
\author{Maxim Sokhatsky}

%HEVEA \begin{divstyle}{selectedwrapper}
%HEVEA \begin{divstyle}{wrapper}

%HEVEA \begin{divstyle}{block}
\section*{Data Consulting}
\paragraph{}
    We are mainly using NoSQL products. In case we need high throughtput
    scalable storages we use Riak. For transactional, query featured, highly
    available cases we use Mnesia. We also maintain our own Amazon Dynamo implementation
     \footahref{https://synrc.github.io/kai/}{KAI} along with our
    \footahref{https://synrc.github.io/kvs/}{KVS} Abstract Database with REST interface.
%HEVEA \end{divstyle}

%HEVEA \begin{divstyle}{block}
\section*{System Architect}
\paragraph{}
    In cases where we need lightweight Erlang solution we use GProc as PubSub messaging.
    For language independent solutions we use RabbitMQ along with our \footahref{https://synrc.github.io/mqs/}{MQS} library.
    In some projects less probably we can use ZMQ.
%HEVEA \end{divstyle}

%HEVEA \begin{divstyle}{block}
\section*{Web Applications}
\paragraph{}
    We used to embed WebMachine/MochiWeb as web server and web framework.
    But currently we have sticked with Cowboy web server.
    For products which require pure TCP listeners and switchable protocols we use Ranch.
    For Web Apps and Binary Push we use \footahref{http://synrc.com/apps/n2o}{N2O}.
%HEVEA \end{divstyle}

%HEVEA \end{divstyle}
%HEVEA \end{divstyle}
%HEVEA \begin{divstyle}{clear}{~}\end{divstyle}

%HEVEA \begin{divstyle}{nonselectedwrapper}
%HEVEA \begin{divstyle}{verywidecol}
\section*{Voxoz Erlang Cloud Control Panel}
\paragraph{}
    Erlang PaaS front end built using N2O, KVS and MQS the key erlang
    applications for Voxoz Xen Cloud.

%HEVEA \rawhtmlinput{templates/voxoz.htx}

\@fontsize{3}{Backend: Maxim Sokhatsky \@br Design: Andrii Zadorozhnii \@br}

%HEVEA \end{divstyle}
%HEVEA \end{divstyle}

%HEVEA \begin{divstyle}{nonselectedwrapper}
%HEVEA \begin{divstyle}{verywidecol}
\section*{Skyline App Store}
\paragraph{}
    Skyline App Store is Voxoz PaaS and Xen ready web shop in Erlang designed as
    promo N2O example application. It uses N2O, KVS, Bootstrap style with
    node js tools for JavaScript minification and LESS assembling.

%HEVEA \rawhtmlinput{templates/skyline.htx}

\@fontsize{3}{Backend: Maxim Sokhatsky \@br Design: Andrii Zadorozhnii \@br}

%HEVEA \end{divstyle}
%HEVEA \end{divstyle}


%HEVEA \begin{divstyle}{nonselectedwrapper}
%HEVEA \begin{divstyle}{verywidecol}
\section*{Scalable Web and Gaming Platform}
\paragraph{}
    For the Turkish audience we created scalable distributed system using Erlang, Riak,
    GProc, Cowboy, AMF, RabbitMQ, Nitrogen, GlusterFS, NGINX and other tools. 
    Please read the technical description
    \footahref{https://speakerdeck.com/5ht/kakaranet-presentation}{Kakaranet Tech}.
    We have implemented Facebook, Paypal, Credit Cards, Mobile and Wired Payments,
    Mail system support and SNMP monitoring tools. We've done full HTML5 optimization
    with A marks for both Yahoo! YSlow and Google Page Speed. Kakaranet is lineary scalable
    Facebook-like application with feeds, subscriptions and messaging along with
    powerful turn-based board games engine. The heart of the system the RabbitMQ
    cluster. The System runs on top of Riak cluster with easy maintenance from Web
    and supports hot code reload using sync and update scripts approach 
    instead of native OTP mechanism despite using native OTP releases. 

%HEVEA \rawhtmlinput{templates/kakaranet.htx}

\@fontsize{3}{Mentor: Maxim Sokhatsky \@br Game Server: Serge Polkovnikov \@br}

\paragraph{}
    Full size of codebase is fit 2.88MB floppy disk and is quite managable yet.
    The system inside is essentially built upon binary protocols AMQP, BER and AMF
    except content delivery to Web clients which is in fact gzipped.

%HEVEA \end{divstyle}
%HEVEA \end{divstyle}


%HEVEA \begin{divstyle}{nonselectedwrapper}
%HEVEA \begin{divstyle}{verywidecol}
\section*{SNMP Monitoring System}
\paragraph{}
    For our servers we created open-source Erlang based SNMP monitoring tools
    with clear and simple interface which allows you to feel the heartbeat of
    your system on any device despite the screen size.

%HEVEA \rawhtmlinput{templates/hal.htx}

\@fontsize{3}Mentor: Andrew Zadorozhny \@br Design: Alex Kalenuk

%HEVEA \end{divstyle}
%HEVEA \end{divstyle}

%HEVEA \begin{divstyle}{nonselectedwrapper}
%HEVEA \begin{divstyle}{verywidecol}
\section*{Rich HTML5 Clients}
\paragraph{}
    For the HTML5 web platform and pure JavaScript client-side we would recommend to use
    Chaplin and Brunch which is agnostic to frameworks and libraries HTML5 application assembler.
    We are also stick with functional languages approach on client side.
    In that way we are using LiveScript a CoffeeScript derivative as main functional language for JavaScript replacement.
    In short: we are using pure HTML5 on client side and pure Erlang on server side.
%HEVEA \end{divstyle}
%HEVEA \end{divstyle}


%HEVEA \begin{divstyle}{nonselectedwrapper}
%HEVEA \begin{divstyle}{verywidecol}
\section*{Binary Protocols}
\paragraph{}
    We are trying to build our products
    and services around single core of technologies. We believe that focusing on understandable
    and manageable, proven in industry technologies will bring significant benefits to all users.
    We have chosen set of technologies that have proved its maturity, clarity and efficiency.
    We oriented on strong and mature binary protocols and data formats such as BER and ASN.1.
    We are not accepting text formats and preffer binary protocols :)

\footahref{apps.htm}{Follow The Source}

%HEVEA \end{divstyle}
%HEVEA \end{divstyle}


%HEVEA \begin{divstyle}{clear}{~}\end{divstyle}


\end{document}
