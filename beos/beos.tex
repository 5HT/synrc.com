\documentclass[11pt]{article}
\usepackage{ifthen}
\usepackage{graphicx}
\usepackage{cite}
\usepackage{hevea}
\input{synrc.hva}
%HEVEA \loadcssfile{../synrc.css}
\begin{document}
%HEVEA \rawhtmlinput{templates/head-hevea.htx}

\title{BeOS page}
\author{Maxim Sokhatsky}

%HEVEA \begin{divstyle}{wrapper}
%HEVEA \begin{divstyle}{threecol}

\section*{\footahref{beos_collection.htm}{Collection}}
%HEVEA \begin{divstyle}{block}
\paragraph{}
Here is BeOS collection archive which includes
BeOS PR1, R4, R4.5, R5, R5.0.3, R5.1 and various documention.
It is mainly for historical reason and as reference platform for binary compatibility.
%HEVEA \end{divstyle}

%HEVEA \end{divstyle}
%HEVEA \begin{divstyle}{widecol}

\section*{Apps}
%HEVEA \begin{divstyle}{block}
\paragraph{}
Here is BeOS app collection that are both compatible with R5 and Haiku.
Almost greater part of BeOS apps has gone so if you have great BeOS app that is
not listed on FTP please feedback.
%HEVEA \end{divstyle}

%HEVEA \end{divstyle}
%HEVEA \begin{divstyle}{threecol last}

\section*{Report Notes}
%HEVEA \begin{divstyle}{block}
\paragraph{}
Here is some articles about BeOS, Haiku and geek computing produced by Synrc Research Center.
%HEVEA \end{divstyle}

Reading: {\footahref{http://kernel-joe.dreamwidth.org}{http://kernel-joe.dreamwidth.org}}

%HEVEA \end{divstyle}
%HEVEA \end{divstyle}
%HEVEA \begin{divstyle}{clear}{~}\end{divstyle}

%HEVEA \footerfalse
%HEVEA \rawhtmlinput{templates/foot.htx}

\end{document}
